\section{Introduzione}

\subsection{Obiettivi}
Lo scopo principale di questo progetto è stato quello di creare un' agente artificiale a supporto dell'applicativo WoodLot.
WoodLot è un e-commerce che permette l'acquisto simbolico di un albero. Gli alberi acquistati dagli utenti saranno piantati da contadini. I contadini, che si possono registrare autonomamente alla piattaforma, possono provenire da tutti i paesi del mondo.
Il compito dell'agente sarà quello di assegnare gli alberi acquistati dagli utenti ai contadini. 
Un contadino dopo aver ricevuto un albero avrà a disposizione, un mese per piantare il seme e communicarlo in piattaforma, l'ordine non evaso sarà assegnato ad un nuovo contadino.
L'assegnazione degli alberi ai contadini deve rispettare alcuni vincoli: 
\begin{itemize}
\item contadino che si trova nel luogo adatto alla crescita dell'albero
\item contadino che ha in cura meno alberi di tutti 
\item contadino che non ricevuto ri-assegnamenti dei suoi ordini nell'ultimo mese
\end{itemize}



\subsection{Specifica PEAS}
\begin{itemize}
\item \textbf{Performance:} sono le misure di prestazione adottate per valutare l’operato di un agente,
in questo caso valutiamo se .
\item \textbf{Environment:} Descrizione degli elementi che formano l’ambiente (descritto sotto).
\item \textbf{Actuators:} Gli attuatori disponibili dell’agente per intraprendere le azioni. In questo caso gli attuatori saranno 
\item \textbf{Sensors:} 
\end{itemize}

\subsection{Caratteristiche dell'ambiente}
Tale problema si è affrontato partendo da un ambente realizzato con le seguenti caratteristiche: 

\subsection{Analisi del problema}
Il problema può essere formalizzato descrivendolo: 
\begin{itemize}
\item \textbf{Stato iniziale:} 
\item \textbf{Descrizione delle possibili azioni:} 
\item \textbf{Modello di transizione:}
\item \textbf{Test obiettivo:} 
\item \textbf{Costo del cammino:} 
\end{itemize}

Per la realizzazione del programma agente è stato deciso di affrontare questo problema implementando non un'unica soluzione, ma diverse soluzioni utilizzando gli algoritmi di ricerca studiati durante il corso di "Fondamenti di Intelligenza Artificiale A.A 2021/22", in modo da valutarne i diversi punti di forza e debolezza. 


\begin{equation} \label{eqRelativity} E=m*c^2 \end{equation}




\begin{table}[h!]
\begin{center}
\caption{Binary Table}
\begin{tabular}{|l|l|l|l|l|}
\hline
0 & 1 & 0 & 1 & 0 \\ 
1 & 0 & 1 & 0 & 1 \\
0 & 1 & 0 & 1 & 0 \\
1 & 0 & 1 & 0 & 1 \\
\hline
\end{tabular}
\end{center}
\end{table}
